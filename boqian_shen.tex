# **Flow Mechanism of Shale Gas**

- Material’s Properties
    - Permeabilit is alwasy relatively low, less than 1 md.
    - The stresssensitive parameters, including organic richness, porosity, thickness and lateral extent, can vary significantly with in situ stress changes.
- **Adsorption/Desorption Mechanism**
    - There are three states of has reserved in shale reservoir: free gas, adsorbed gas(main state 20-80%) and dissolved gas.
    - With the decrease in pressure, adsorbed gas sill become free gas in the early period of exploitation.
    - methane adsorption
    - adsorption capacity (TOC total organic carbon, shale contains high orgnic matter→ high gas adsorption)
    - gas adsorption process can be described by various isotherm models: Langmuir’s type model (simplest and effective), Freundlich-type model, Langmuir–Freundlich-type model, D-R-type model, BET-type model and Toth-type models.
    - Multi-component adsorption mechanisms,composition of varieties of hydrocarbons (C2+) and subsequently high total organic content
    (TOC)
- **Flow Mechanisms of Gas Transport in Shale Gas Reservoir**
    - Flow Regime
        - pore size ranging from macroscale (>1mm) to nanoscale (<100 nm)
        - Kn → flow regime → governing equation+boundary condition
    - Based on the Knudsen numbers (Kn), the flow mechanism can be classified as two categories:
        - molecular dynamics for shale gas transportation: adsorption, diffusion, displacement and other mechanisms
        - lattice Boltzmann method : slippage, Knudsen diffusion, and apparent permeability correction(to improve Darcy’s law where matrix with low velocities and freactures with high velocities.
        - Apparent Permeability Correction
        - Improved Darcy Model in Fractures

# **Reservoir Models (reservoir-scale Gas flow  simulation)**

- Flow model
    - porous media: inter-particle and intra-particle(organic matter pores within kerogen and mineral particles) pores
    - Kerogen is the place gas adsorbs on the wall and dissolves within it
    - Gas is considered as wetting phase
    - Four types of organic matter: hydraulic fractures, natural fractures,
    kerogen (organic matrix) and inorganic matter, with the derease in pore size
    - Another classification: organic porosity, inorganic porosity, natural fractures and hydraulic fractures
    - single porosity for shale gas simulation: Multiple interacting continua (MINC) and explicit fracture modeling
    - for different pore sizes such as dual porosity model, connection between nanopores and micropores in organic matter has to be considered: Darcy’s law (limited, low permeability) and Fickian diffusion should be considered
- FlowModel Coupled with Geomechanics
    - A popular trend is to use geomechanism with flow model for gas production simulation, otherwise the production will be overestimated, since the production is sensitive to fracture aperture changes
    - pressure, deformation
    - linear elasticity, nonlinear elasticity(for better estimation on permeability)

# **Numerical Approaches**

- Finite Difference Method (FDM)
    
    efficiency and simplicity, with rectangular and triangular grid, uniform and non-uniform meshes, cartesian and curvilinear coordinates, 1D to 3D.  Firoozabadi 1999
    
- Finite Element Method (FEM)
    
    more arrutate, unstructured meshes (fracture)
    
    Multiscale mixed finte element (MsMFE) method 
    
    The main process is to make this method as geometrically flexible as possible and developing coarsening strategies that semiautomatically adapt to barriers, channels, faults and wells in a way that ensures good accuracy for a chosen level of coarsening.
    
    AMG
    
- Finite Volume method FVM
    
    more easily implemented with unstructured grids
    
- Fast marching method (FMM)